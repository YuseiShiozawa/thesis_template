\section{実験概要}
本研究で作成したドキュメントの有用性を確認するため, 津田沼チャレンジ\cite{tsudachare}のコースを用いて自律移動実験を行った. 
実験では, ドキュメントによる手順通りにナビゲーションのパラメータを調整されたロボットが, 自律的に設定されたコースを完走できるかで, ドキュメントの有用性を確認することを目的とした. 

検証のため以下の2つの実験を行った. 
\begin{table}[H]
     \centering
     \begin{tabular}{ll}
         $実験1$ & ドキュメント作成者が2025年度版コースを対象として行った有用性の検証 \\
         $実験2$ & ROS初心者の学部3年生が2024年度版コースを対象として行った有用性の検証 \\
     \end{tabular}
\end{table}
%実験は, ドキュメント作成者が2025年度版コースで行ったものと, ROS初心者の学部3年生がドキュメントを参照して2024年度版コースで行ったものの2種類である. 
それぞれのコースを\figref{Fig:Course map of the Tsudanuma Challenge 2025}, \figref{Fig:Course map of the Tsudanuma Challenge 2024}に示す.
実験1では, 走行地図の作成から自己位置推定および経路計画に関する各種パラメータの調整までを全て行った. 
また, 2025年8月27日に行われた実験1の本走行においては, LiDARの計測値が一定の閾値以下になったときに減速, 停止を行う処理や, 
特定の場所で一時的にコストマップを無効化する処理, ロボットがスタックしたときにコストマップを初期化する処理といった補助的なノードを併用した. 
これらは, 第5章にて述べる, 人やロボットなどの動的障害物が多いつくばチャレンジ\cite{つくばチャレンジ}を見据えた自律移動上の拡張機能として導入した. 
一方, 実験2では, 作成者がまとめたドキュメントを基に, 作成者が作成した地図を使用してナビゲーションのパラメータ調整を行った. 
また, パラメータ調整に要した時間についても記録し, 調整の負担を把握する参考とした. 

実験には, 本研究室で開発されているロボットORNE-box2\cite{井口颯人2023屋外自律移動ロボットプラットフォーム-orne}を使用した. 
その外観を\figref{Fig:ORNE-box2}に示す. 
また, ナビゲーションにはROS Navigation stackを用いた. ORNE-box2を含めたそのシステム構成を\figref{Fig:system}に示す. 
ORNE-box2はPCとしてJetson AGX Xavier (32GB RAM)を搭載しており, 
3D-LiDAR (R-Fans-16), 2D-LiDAR (UTM-30LX-EW), IMU (ADIS16470), およびエンコーダを備えている. 
2D-LiDARは障害物検出に使用し, 3D-LiDARは自己位置推定に用いた. 
IMUとエンコーダに基づくオドメトリはジャイロオドメトリとして統合し, AMCLに対して移動情報を提供している. 
ナビゲーションシステムは, Map Serverが提供する静的地図を基にAMCLが自己位置を推定し, Move Baseが経路計画を行い, 速度指令値を送信する構成である. 
また, Waypoint Serverは, あらかじめ設定した複数の目的地を管理し, 各地点に到達するごとにMove Baseに対して次の目的地を送信することで, 
指定されたコースでの自律移動を実現している. 
\begin{figure}[hbtp]
  \centering
 \includegraphics[keepaspectratio, scale=0.5]
      {images/2025tsudacha.png}
 \caption{Course map of the Tsudanuma Challenge 2025 (souce: \cite{tsudachare})}
 \label{Fig:Course map of the Tsudanuma Challenge 2025}
\end{figure}
\begin{figure}[hbtp]
  \centering
 \includegraphics[keepaspectratio, scale=0.5]
      {images/2024tsudacha.png}
 \caption{Course map of the Tsudanuma Challenge 2024 (souce: \cite{tsudachare})}
 \label{Fig:Course map of the Tsudanuma Challenge 2024}
\end{figure}
\begin{figure}[hbtp]
  \centering
 \includegraphics[keepaspectratio, scale=0.4]
      {images/box2.png}
 \caption{ORNE-box2}
 \label{Fig:ORNE-box2}
\end{figure}
\begin{figure}[H]
  \centering
 \includegraphics[keepaspectratio, scale=0.11]
      {images/systembox.png}
 \caption{System configuration}
 \label{Fig:system}
\end{figure}




\newpage
\section{実験結果}
\subsection{実験1}
\begin{table}[H]
  \centering
  \caption{Navigation results of the Tsudanuma Challenge 2025}
  \label{tab:tsudanuma_result}
  \begin{tabular}{lcc}
    \hline
    \textbf{Run type} & \textbf{Number of trials} & \textbf{Number of success} \\
    \hline
    Preliminary run & 10 & 10 \\
    Official run        & 1  & 1  \\
    \hline
  \end{tabular}
\end{table}
\tabref{tab:tsudanuma_result}に示すように, 調整後のシステムを用いて事前走行を10回実施した結果, 
すべての試行でゴールまでの完走を確認した. 
走行コースの全長は約1799m, 走行時間はおよそ40分であり, すべての走行で同区間を完走した. 

さらに, 2025年8月27日に行われた津田沼チャレンジの本走行においてもコースを完走し, 自己位置の破綻や経路追従の失敗は見られなかった. 
本走行での実験風景を\figref{Fig:tsudanumaaa}に示す. 
\begin{figure}[hbtp]
  \centering
 \includegraphics[keepaspectratio, scale=0.2]
      {images/tsudanumachalle.png}
 \caption{Experiment scene}
 \label{Fig:tsudanumaaa}
\end{figure}

\subsection{実験2}
\begin{table}[htbp]
  \centering
  \caption{Navigation results of B3 students in the Tsudanuma Challenge 2024}
  \label{tab:b3_results}
  \begin{tabular}{lccc}
    \hline
     & \textbf{No.1} & \textbf{No.2} & \textbf{No.3} \\
    \hline
    Success & Yes & Yes & Yes \\
    Total time to trial [h] & 5.0 & 3.5 & 4.5 \\
    \hline
  \end{tabular}
\end{table}
\tabref{tab:b3_results}に示すように, ドキュメントの手順に沿ってパラメータ調整を行った結果, 3名ともコースの完走を達成することができた. 
走行コースの全長は約1661m, 走行時間はおよそ35分であり, すべての走行で同区間を完走した. 
いずれの実験においても, ロボットは経路から大きく逸脱することなく目標地点まで走行でき, 安定したナビゲーション動作が確認された. 
また, パラメータ調整に要した時間は3.5〜5.0時間であり, いずれの参加者も作業を完遂することができた. 
これにより, ドキュメントの有用性が数例ではあるが確認された. 


