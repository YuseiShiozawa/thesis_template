%!TEX root = ../thesis.tex
\chapter*{概要}
\thispagestyle{empty}
%
\begin{center}
  \scalebox{1.5}{ROSベースの自律移動ロボットにおける}\\
  \scalebox{1.5}{ナビゲーション調整手順の体系化}\\
\end{center}
\vspace{1.0zh}
%

本論文では, ROS(Robot Operating System)をベースとした自律移動ロボットにおけるナビゲーション調整手順の体系化について述べる. 
近年, ROSベースの自律移動ロボットのナビゲーション技術の活用が進んでいるが, ROS Navigation stackを用いた地図ベースの自己位置推定, 経路計画に基づく自律移動を行うためには複数のパラメータを適切に調整する必要がある. 
しかし, 現状ではパラメータ調整に関する明確な指針はあまり示されておらず, 特に屋外環境に関する情報は少ない. 
本研究では, ナビゲーションの調整手順をドキュメントとしてまとめ, 調整方法の一例を示すことを目的とする. 
作成したドキュメントの有効性を検証するため, ROS Navigation stackを使用し, 千葉工業大学津田沼キャンパスで実施される技術チャレンジである
津田沼チャレンジのコースで自律走行実験を行った. その結果, 事前走行および本走行においてコースを完走することができた. 
さらに, ROS初心者の学部3年生3名が本ドキュメントを用いてナビゲーション調整を行った結果, 全員がコースの完走を実現し, 本ドキュメントの妥当性が確認された. 

\vspace{1.0zh}

キーワード: 自律移動ロボット, ナビゲーション
%
\newpage
%%
\chapter*{abstract}
\thispagestyle{empty}
%
\begin{center}
  \scalebox{1.3}{Systematization of Navigation Adjustment Procedures}
  \scalebox{1.3}{for ROS-Based Autonomous Mobile Robots}
\end{center}
\vspace{1.0zh}
%
This paper describes the systematization of navigation tuning procedures for autonomous mobile robots based on the Robot Operating System (ROS). 
In recent years, the utilization of navigation technologies for ROS-based autonomous mobile robots has advanced. However, to perform autonomous navigation based on map-based self-localization and path planning using the ROS Navigation stack, multiple parameters must be appropriately tuned. 
However, clear guidelines for parameter tuning are currently scarce, with particularly limited information available for outdoor environments. 
This research aims to document the navigation tuning procedure and present an example of the tuning methodology. 
To validate the effectiveness of the created documentation, autonomous navigation experiments were conducted on the course of the Tsudanuma Challenge, a technical challenge held at Chiba Institute of Technology's Tsudanuma Campus, using the ROS Navigation stack. 
As a result, the vehicle successfully completed the course during both preliminary and official runs. 
Furthermore, three third-year undergraduate students new to ROS used this document to perform navigation tuning. All three successfully completed the course, confirming the validity of this document. 


\vspace{1.0zh}

keywords: Autonomous Mobile Robot, Navigation
