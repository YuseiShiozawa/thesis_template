%!TEX root = ../thesis.tex
\chapter*{概要}
\thispagestyle{empty}
%
\begin{center}
  \scalebox{1.5}{ROSベースの自律移動ロボットにおける}\\
  \scalebox{1.5}{ナビゲーション調整手順の体系化}\\
\end{center}
\vspace{1.0zh}
%

本論文では, ROS(Robot Operating System)をベースとした自律移動ロボットにおけるナビゲーション調整手順の体系化について述べる. 
近年, ROSベースの自律移動ロボットのナビゲーション技術の活用が進んでいるが, 自律移動を行うためには複数のパラメータを適切に調整する必要がある. 
しかし, 現状ではパラメータ調整に関する明確な指針はあまり示されておらず, 特に屋外環境や実ロボットに特化した情報は少ない. 
本研究では, ロボットにおけるナビゲーションの調整手順をドキュメントとしてまとめ, 適切な調整方法を明らかにすることを目的とする. 
作成したドキュメントの有効性を検証するため, ROS Navigation stackを使用し, 千葉工業大学津田沼キャンパスで実施される技術チャレンジである
津田沼チャレンジのコースで自律走行実験を行った. その結果, 事前走行および本走行においてコースを完走することができた. 
さらに, ROS初心者の学部3年生3名が本ドキュメントを用いてナビゲーション調整を行った結果, 全員がコースの完走を実現し, 本ドキュメントの有効性が確認された. 

\vspace{1.0zh}

キーワード: 自律移動ロボット, ナビゲーション
%
\newpage
%%
\chapter*{abstract}
\thispagestyle{empty}
%
\begin{center}
  \scalebox{1.3}{Systematization of Navigation Adjustment Procedures}
  \scalebox{1.3}{for ROS-Based Autonomous Mobile Robots}
\end{center}
\vspace{1.0zh}
%
This paper describes the systematization of navigation tuning procedures for autonomous mobile robots based on the Robot Operating System(ROS). 
In recent years, the utilization of navigation technologies for ROS-based autonomous mobile robots has been increasing. 
However, achieving autonomous navigation requires appropriate tuning of multiple parameters. 
At present, there are few clear guidelines for parameter tuning, particularly for outdoor environments or real robot implementations. 
The objective of this study is to compile navigation tuning procedures into documentation and clarify appropriate adjustment methods. 
To evaluate the effectiveness of the proposed documentation, autonomous navigation experiments were conducted using the ROS Navigation Stack on the Tsudanuma Challenge course at Chiba Institute of Technology. 
As a result, the robot successfully completed both the preliminary and main runs of the course. 
Furthermore, three third-year undergraduate students who were beginners in ROS used the documentation for navigation tuning, and all of them achieved successful completion of the course, 
hereby confirming the effectiveness of the documentation. 

\vspace{1.0zh}

keywords: Autonomous Mobile Robot, Navigation
