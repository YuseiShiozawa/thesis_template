\section{ドキュメントの構成}
作成したドキュメントの構成は以下の要素から構成されている. 
また, 作成にあたってROS WikiのNavigationページを参考にした.
\cite{ros_wiki_navigation}
\begin{itemize}
     \item \textbf{自己位置推定(AMCL)}
     \begin{itemize}
        \item \textbf{最低限の設定}
        \item \textbf{主要パラメータの調整}
        \item \textbf{その他パラメータの調整}
        \item \textbf{ROS\_ERRORが出たときの問題と対処法}
    \end{itemize}
     \item \textbf{経路計画(Move Base)}
    \begin{itemize}
        \item \textbf{最低限の設定}
        \item \textbf{Move Baseの土台となるパラメータ調整}
        \item \textbf{リカバリ動作のパラメータ調整}
        \item \textbf{コストマップのパラメータ調整}
        \item \textbf{ローカルプランナーのパラメータ調整}
        \item \textbf{グローバルプランナーのパラメータ調整}
        \item \textbf{ROS\_WARNが出たときの問題と対処法}
    \end{itemize}
\newpage
     \item \textbf{地図}
    \begin{itemize}
        \item \textbf{地図作成方法の説明}
        \item \textbf{slam\_toolboxでの地図作成方法}
        \item \textbf{glimでの地図作成方法}
    \end{itemize}
    \item \textbf{オドメトリ}
    \begin{itemize}
        \item \textbf{オドメトリの調整手順}
    \end{itemize}     
\end{itemize}

\section{ドキュメントの例示}
本論文では, 作成したドキュメントの中から例示として, オドメトリ調整と自己位置推定(AMCL)の
2項目を取り上げ, 記載内容の一部を示す. 

\subsection{オドメトリ調整}
調整対象のパラメータは, 車輪半径に関する補正係数であるwheel\_radius\_multiplierと, 車輪環距離に関する補正係数であるwheel\_separation\_multiplierの2つである. 
これらはロボットのオドメトリの正確さに直結するため, 自己位置推定を行う上で最初に調整すべき項目である. 



